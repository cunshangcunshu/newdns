\documentclass[11pt]{article}

\usepackage{fullpage}
\usepackage{makeidx}
\usepackage{xspace}
\usepackage[colorlinks=true, linkcolor=blue]{hyperref}
\usepackage{color}

\makeindex

\begin{document}


%
% Internal Commands
%
\newcommand*{\dnsc}[0]{\texttt{dns.c}\xspace}
\newcommand*{\lref}[1]{\hyperref[#1]{#1}\index{#1}}
\newcommand*{\dn}[1]{\texttt{#1}\xspace}

\newenvironment{todo}[1]{
	\begin{color}{red}
	\sf
	\begin{tabular}{|p{0.9\textwidth}|}
	\hline
	TODO---\uppercase{#1} \\\hline
	\index{TODO!\lowercase{#1}}
}{
	\\\hline
	\end{tabular}
	\end{color}
}

%
% Title page
%
\title{\dnsc Userguide [draft]}
\date{May 2010}
\author{William Ahern}
\maketitle

\tableofcontents

\pagenumbering{roman}

\clearpage

\setcounter{page}{1}
\pagenumbering{arabic}


\section{Introduction}

\dnsc is an asynchronous, reentrant, recursive, embeddable DNS resolver.
There are many resolver libraries available, including
\href{http://www.chiark.greenend.org.uk/~ian/adns/}{\textit{adns}},
\href{http://c-ares.haxx.se/}{\textit{c-ares}},
\href{http://www.corpit.ru/mjt/udns.html}{\textit{udns}}, and
\href{http://nlnetlabs.nl/projects/ldns/}{\textit{ldns}}. Though often well
written, none have yet to meet my personal requirements, derived after years
of writing DNS--intensive, non-blocking server software.

\subsection{Asynchronous}

Asynchronous means that \dnsc does not block on network I/O.\footnote{The
core resolver does not do file I/O; however, some optional, ancillary
routines use \texttt{stdio}.} Internally \dnsc directly uses the \textsc{BSD
Sockets} API for network I/O. Whenever a socket operation would block, the
\textsc{BSD Sockets} routine immediately returns, setting
\texttt{EAGAIN}\index{errno!EAGAIN}\footnote{\dnsc translates \texttt{EINPROGRESS}\index{errno!EINPROGRESS}
and \texttt{EWOULDBLOCK}\index{errno!EWOULDBLOCK} to \texttt{EAGAIN}.}.
This error value is returned to the calling application with haste.

\dnsc does not presume any particular kind of event loop\footnote{However,
\texttt{events()} can be configured to return \textit{libevent} event sets},
nor queues events internally. Instead, resolver objects expose a common set
of interfaces: \texttt{pollfd()}, \texttt{events()}, \texttt{elapsed()}.

\subsection{Reentrant}

\dnsc rigorously adheres to an object--oriented design pattern. There is no
global, mutable data. Some types of objects can be shared between resolver
objects, but once instantiated and configured they're immutable. For
example, while two resolver objects can share a \texttt{struct dns\_hints}
object, to query the hints the caller must pass a separate iteration state.

\hfill

\begin{todo}{atomics}
Most objects use reference counting for
lifetime management. These reference counts are intended to be incremented
and decremented by async--signal safe methods, but as these are inherently
non--portable, and because there is no easy way to detect compiler atomic
intrinsics, currently a \texttt{volatile sig\_atomic\_t} type is used with
increment and decrement operators. In the future atomic intrinsics should be
specifiable at build--time, similar to \texttt{DNS\_RANDOM}.
\end{todo}

%\begin{color}{red}
%\sf
%\begin{tabular}{|p{0.9\textwidth}|}
%\hline
%TODO---ATOMICS \\\hline
%\index{TODO!atomics}Most objects use reference counting for
%lifetime management. These reference counts are intended to be incremented
%and decremented by async--signal safe methods, but as these are inherently
%non--portable, and because there is no easy way to detect compiler atomic
%intrinsics, currently a \texttt{volatile sig\_atomic\_t} type is used with
%increment and decrement operators. In the future atomic intrinsics should be
%specifiable at build--time, similar to \texttt{DNS\_RANDOM}. \\\hline
%\end{tabular}
%\end{color}

\subsection{Recursive}

Most client DNS libraries operate in stub mode. A stub resolver acts as a
simple messenger between the client application and a \lref{recursive
resolver} on the local network. \hyperref[stub resolver]{Stub
resolvers}\index{stub resolver} require minimal sophistication, relaying a
single DNS question and waiting for a single reply. Traditionally, the
presumption has been that most client applications merely want to resolve a
hostname to an IP address via an A or AAAA record, and this minimal
sophistication fully sufficed to address this need.

This presumption has proven costly as SRV and other similar record types
have become widespread. The reason being, stub resolvers usually have no
ability to follow the embedded canonical names in the record data of these
types. For example, an MX record specifies a canonical host name for which
an IP address should be queried. Usually the A or AAAA record for this
canonical host name is included as \lref{glue} in the original answer, but
this is not guaranteed. Indeed, \lref{out--of--bailiwick} host names---names
which are not in the same lexical zone of the nameserver, and thus for which
the nameserver is unlikely to be authoritative or able to provide nameserver
glue for the authoritative server---are becoming increasingly common.

Unfortunately, the burden of following these canonical names inevitably
falls to the client application. First, as mentioned before stub resolvers
usually do not have the sophistication to recursively resolve these names.
Second, recursive nameservers rarely resolve these canonical names. Rather,
the role of a recursive nameserver is to follow chains of authority and
CNAME references, not to understand the semantics and data dependencies of
myriad other record types.

Historically, the only client applications required to implement this logic
have been mail servers. The move to SRV, NAPTR, and similar record types,
however, dramatically increases the categories of client applications
required to support this logic. Clearly, a reusable implementation would be
desirable for these new classes of applications.

Broadly speaking there are two ways to go about supporting this canonical
name problem. The easiest would seem to be to hack or wrap existing stub
resolvers to provide the minimally necessary functionality. Alternatively,
one could write a full--blown recursive query implementation, which
inherently provides the lesser capability. \dnsc takes this second approach,
and thus can operate in a fully recursive query mode, a traditional stub
mode, or a ``smart'' stub, minimally recursive mode. The result is that
\dnsc can follow canonical names for known record types, and is easily
modifiable---and readily composable---to deliver all relevant DNS
information to a requesting application in one simple operation.

\subsection{Embeddable}

\dnsc is designed to have zero build--time or run--time dependencies,
requiring only a nominally \textsc{POSIX}\index{POSIX} compliant
environment. The header and source files, \texttt{dns.h} and \texttt{dns.c},
compile to a single object file. This object file can be built into the
application directly, or optionally installed as a static or dynamic
library. All macros and routines use a ``dns\_'' prefix, and all non--public
routines are declared \texttt{static}.

The code adheres to the ISO C99\index{C99} standard, with the notable exception
that the packet structure relies on an anonymous union\footnote{Anonymous
structures and unions are scheduled for inclusion in the next ISO C
standard.}. All compilers known by this author support anonymous unions.

Furthermore, the license\index{license} has been deliberately chosen to
allow integration into all classes of applications, not only those licensed
as Free Software or Open Source. Niche, novel DNS applications tend to occur
in commercial and private software with quirky build, execution, and
maintenance environments. Allowing unrestricted embedding supports the goal
of providing a simple, easily integrated resolver implementation.

\section{Configuration}

\subsection{\texttt{/etc/resolv.conf}}

\subsection{\texttt{/etc/hosts}}


\section{Glossary}

\begin{description}
\newcommand{\term}[1]{\item[#1]\index{#1}\label{#1}}

\term{glue} packet which resolves a reference so that the resolver does not
need to execute a separate query. In the case of NS records, especially,
glue might be necessary to prevent an infinite loop. For example, when
querying for ``\texttt{www.example.com. IN A}'' the .com nameserver may
refer the resolver to \dn{ns.example.com.}, the authoritative nameserver for
example.com domains. But in resolving the IP address to
\dn{ns.example.com.}---to which the query for www.example.com would be
sent---the .com nameserver may again refer the resolver to
\dn{ns.example.com.} The solution is for the .com nameserver to include a
glue A or AAAA record for \dn{ns.example.com.}, short--circuiting the
process.

\term{out--of--bailiwick} Describes a domain name which is not in the same
lexical zone of a reference domain name. For example, in the record
``\texttt{example.com. IN MX 10 aspmx.l.google.com.}'' aspmx.google.com. is
not within the lexical zone of example.com. Typically such a response will
not include glue, necessitating an extra recursive query---in this case an A
or AAAA query for aspmx.l.google.com. Additionally, implementations will
often ignore or fail to cache glue records for out--of--bailiwick references
because of security concerns. Out--of--bailiwick is more commonly used in
the context of CNAME references.

\term{recursive resolver} A resolver implementation that is capable of
traversing the hierarchy of global nameserver authorities, starting with the
root zone. A recursive resolver must already know the IP addresses of the
root nameservers. Recursive resolvers are usually stand--alone
implementations residing on the local network.

\term{stub resolver} A resolver implementation which depends on a
\lref{recursive resolver} to fetch answers from the global domain name
system. A stub resolver is usually implemented as a library, communicating
with a recursive resolver over the network on behalf of the client
application. To function a stub resolver must acquire the IP address of
local recursive nameservers by some other means. Traditionally, local
recursive nameservers are listed in \textit{/etc/resolv.conf}, or if
unspecified presumed to be at \textit{127.0.0.1}. Other protocols such as
\textit{Multicast DNS} and \textit{DHCP} provide methods to identify local recursive
nameservers.

\end{description}

\printindex

\end{document}
